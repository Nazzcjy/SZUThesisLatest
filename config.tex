%---------------------------------------------------------------------------%
%->> Cover information 封面信息
%---------------------------------------------------------------------------%
\classid{\ TP39\;}% 分类号
\udc{\ \ \ 004\ \ }
\confidential{\ \ \ \ \ 公开}% 密级
%- 注:\title包含两个参数
% \title{深圳大学\LaTeX{}模板}{}% 单行题目,第二个参数为空
\title{xxxxxxxxxxx}{yyyyyyyyyyyyyyyy}% 多行题目
%- 注:英文题目用于生成Abstract的页眉,只有一个参数
\TITLE{aaaaaaaaaaaaaaaaaaaaaaaaaaaaaaaaaaaaaaaaaaaaaaaaaaaaaaa}
\author{\ \ \ \ \ \ \ \ xxx}% 论文作者
\idnumber{\ \ \ \ \ \ \ \ xxxxxxxxx}
\major{\ \ \ \ \ \ \ \ 计算机科学与技术}% 学科专业名称
\dtype{\ \ \ \ \ \ \ \ 工学}% 学科门类名称
%- 注:以下两个类型支持多行
\institute{\ \ \ \ \ \ \ \ 计算机与软件学院}% 院系名称单行
% \institute{某某学院\\某某实验室}% 院系名称多行
\advisor{\ \ \ \ \ \ \ \ xxxxxxxxxxx}% 指导教师单行
% \advisor{张老师\ 教授\\王老师\ 研究员}% 指导教师多行
% !!! 记得切换学硕专硕
\DEGREE{MasterXS}% 学术硕士
% \DEGREE{MasterZY}% 专业硕士
%---------------------------------------------------------------------------%
%->> other config
%---------------------------------------------------------------------------%
%- 添加两个命令方便输出
\DeclareRobustCommand\cs[1]{\texttt{\char`\\#1}}
\providecommand\pkg[1]{{\sffamily#1}}
%-
\addbibresource{Biblio/ref.bib}% 参考文献路径
\setlength\bibitemsep{0.0ex plus 0.2ex minus 0.2ex}% set distance between bib entrie
%-
\setcounter{tocdepth}{3}% depth for the table of contents,设为2可不显示subsubsection
\setcounter{secnumdepth}{3}% depth for section numbering, default is 2
%-
%- 某些小语种会超出版面边界,提示Overfull \hbox{}...,中英日韩无需使用(或使用宏包microtype)
% \setlength\emergencystretch{1em}
%-
%- 重新设置 equation, figure, table 的序号
%\numberwithin{equation}{section}% set enumeration level
%\renewcommand{\theequation}{\thesection\arabic{equation}}% configure the label style
%\numberwithin{figure}{section}% set enumeration level
%\renewcommand{\thefigure}{\thesection\arabic{figure}}% configure the label style
%\numberwithin{table}{section}% set enumeration level
%\renewcommand{\thetable}{\thesection\arabic{table}}% configure the label style
\counterwithout{footnote}{chapter}% footnote编号全局连续
%-
%---------------------------------------------------------------------------%
%->> Package
%---------------------------------------------------------------------------%
% -> szuthesis.cls中已经导入的包
% - etoolbox, a toolbox of programming facilities
% - geometry, for layout
% - expl3, LaTeX3 programming environment
% - array
% - ulem, underline
% - xeCJKfntef, underline for CJK
% - fancyhdr, header and footer
% - biblatex
%-


\usepackage{graphicx}
\DeclareGraphicsExtensions{.pdf,.jpg,.png,.eps,.tif,.bmp}% 默认图片格式
\graphicspath{{Image/}}% 默认图片检索路径
%-
\usepackage[format=plain,hangindent=2.0em,font={small},skip=8pt,labelsep=space]{caption}
%-
\usepackage{subcaption}% 处理子图
%-
% \usepackage[list=off]{bicaption} % 双语caption
% \DeclareCaptionOption{bi-second}[]{
%     \def\tablename{Table}%
%     \def\figurename{Figure}%
% }
% \captionsetup[bi-second]{bi-second}
%-
\usepackage[section]{placeins}% 阻止图片浮动超出当前section
%-
\usepackage{enumitem}% 列表环境功能提升
\setlist{nosep}% 默认文本行间距
% \setlist[enumerate]{wide=\parindent}% 是否悬挂对齐,不建议全局修改
% \setlist[itemize]{wide=\parindent}
%-
% \usepackage{verbatim}
%-
% \usepackage{chemfig}% draw 2D chemical structures
% \usepackage[version=4]{mhchem}% typeset chemical formulae [mhchem|chemformula]
%-
% \usepackage{microtype}% improves general appearance of the text, 启用后降低编译效率
%-
% \usepackage{pdflscape}% landscape environment, \begin{landscape} ... \end{landscape}
%-
% \usepackage[usenames,dvipsnames,svgnames,table]{xcolor}% color support
%-
% \usepackage{tikz}% automatically load pgf package, plot with tex
% \usetikzlibrary{positioning, arrows, calc, trees }%
%-
\usepackage{booktabs}% 三线表
%-
\usepackage{listings}% 代码片段
\def\lstlistingname{代码}
\lstset{%
    basicstyle=\linespread{1.2}\small, % 字体
    breaklines=true,                   % 自动换行
    frame=lines,                       % 上下的边框,可选none|single|shadowbox等
    keepspaces=true,
    showstringspaces=false,            % string的空格添加标记,defaul:true
    tabsize=2,                         % tab长度
    % stringstyle=\color{DarkViolet},
    % backgroundcolor=\color{gray!10},
    % commentstyle=\color{ForestGreen},
    % keywordstyle=\color{blue},
}
%-
%%%%%%%%%%%%%%%%%%%%%%%%%%%%%%%%%%%%%%%%%%%%%%%%%%%%%%%%%%%%%%%%%%%%%%%%%%%%%%%%%%%%%%%%%%%%%%%%%%%%%%%%%%%%%%%%
% 自用package建议注释掉没有用的包,有些包会产生冲突,但不建议全部注释,有一些可能是通用的。如果不知道要不要注释可以全部注释重新添加需要的包,或者全部保留编译有问题再排除。
\usepackage{multirow}
\usepackage{hyperref}
\usepackage{makecell}
\usepackage{adjustbox}
\usepackage{amsmath,amssymb,amsfonts}
\usepackage{array}
\usepackage{booktabs}
\usepackage{longtable}
\usepackage{tabularx}
\usepackage{algorithmic}
\usepackage{diagbox}
%%%%%%%%%%%%%%%%%%%%%%%%%%%%%%%%%%%%%%%%%%%%%%%%%%%%%%%%%%%%%%%%%%%%%%%%%%%%%%%%%%%%%%%%%%%%%%%%%%%%%%%%%%%%%%%%

\usepackage[ruled,vlined,linesnumbered]{algorithm2e} % 算法描述
\SetAlgorithmName{算法}{算法}{}
\SetArgSty{textit}
%---------------------------------------------------------------------------%
%->> 配置数学环境
%---------------------------------------------------------------------------%

\usepackage{pifont}
\newcommand*{\dif}{\mathop{}\!\mathrm{d}}
%- 符号表,参考 http://milde.users.sourceforge.net/LUCR/Math/mathpackages/amssymb-symbols.pdf
\usepackage{amsthm} % 定理引理等环境
\theoremstyle{plain}% for theorems, lemmas, propositions, etc
\newtheorem{theorem}              {定理} [chapter]
\newtheorem{axiom}      [theorem] {公理}
\newtheorem{lemma}      [theorem] {引理}
\newtheorem{corollary}  [theorem] {推论}
\newtheorem{assertion}  [theorem] {断言}
\newtheorem{proposition}[theorem] {命题}
\newtheorem{conjecture} [theorem] {猜想}
\newtheorem{assumption} [theorem] {假设}
\theoremstyle{definition}% for definitions and examples
\newtheorem{definition}           {定义} [chapter]
\newtheorem{example}              {例}   [chapter]
\newtheorem{problem}              {问题} [chapter]
\newtheorem{exercise}             {练习} [chapter]
\theoremstyle{remark}% for remarks and notes
\newtheorem*{remark}              {注}
\newtheorem*{solution}            {解}
% \usepackage{mathtools}
\usepackage{unicode-math}
%- 注:unicode-math可以配置数学公式字体,注意包冲突!
%- 已知可能存在冲突的包:amscd,amsfonts,bbm,bm,eucal,eufrak,mathrsfs
\setmathfont{XITSMath-Regular}[
    Extension=.otf, BoldFont=XITSMath-Bold, Ligatures=TeX, StylisticSet = 1,
]
\setmathfont{XITSMath-Regular}[
    Extension=.otf, range={scr,bfscr}, Ligatures=TeX, StylisticSet = 2,
]
\setmathfont{XITSMath-Regular}[
    Extension=.otf, range={cal,bfcal}, Ligatures=TeX, StylisticSet = 1,
]
% \setmathfont{XITS Math Bold}[version=bold]% for bold version % 不兼容StylisticSet=2
% \newenvironment{szumathbf}{\bfseries\mathversion{bold}}{}
\def\XITSMathFontOptions{
    Extension=.otf, BoldFont=XITSMath-Bold, Ligatures=TeX, StylisticSet = 1
}
\setmathrm{XITSMath-Regular}[\XITSMathFontOptions]
\setmathsf{XITSMath-Regular}[\XITSMathFontOptions]
\setmathtt{XITSMath-Regular}[\XITSMathFontOptions]
%-
\def\boldsymbol#1{\symbfit{#1}}
\providecommand{\Vector}[1]{\symbfit{#1}}
\providecommand{\Matrix}[1]{\symbfit{#1}}
\providecommand{\Tensor}[1]{\symbfit{#1}}
\providecommand{\Dif}{\symrm{d}}
\providecommand{\Const}[1]{\symrm{#1}}
\providecommand{\deltarm}{\symrm{\delta}}
\providecommand{\Div}{\operatorname{div}}
\providecommand{\Trace}{\operatorname{tr}}
%---------------------------------------------------------------------------%
%->> 链接,生成书签,在最后
%---------------------------------------------------------------------------%
\usepackage{hyperref}% 超链接,生成书签,[注:放在最后]
\hypersetup{% set hyperlinks
    pdfencoding=auto,% allows non-Latin based languages in bookmarks
    psdextra=true,% extra support for math symbols in bookmarks
    bookmarksnumbered=true,% put section numbers in bookmarks
    pdftitle={\szutitle},% title
    pdfauthor={\szuauthor},% author
    pdfsubject={\szutitle},% subject
    pdfstartview={FitH},% fits the width of the page to the window
    % colorlinks=true,% false: boxed links; true: colored links
    % linkcolor=black,% color of internal links
    % citecolor=blue,% color of links to bibliography
    % filecolor=blue,% color of file links
    % urlcolor=blue,% color of external links
    hidelinks,% hide links color and box
}
%---------------------------------------------------------------------------%
%->> END
%---------------------------------------------------------------------------%