\chapter[附录]{附录A\quad IETM的吉布斯采样公式推导}

本附录将为IETM的吉布斯采样提供推导细节。IETM模型的联合分布函数如公式(\ref{joint1})所示:
\begin{align}
\label{joint1}
p(\mathcal{D},\boldsymbol{\vec l}|\vec\alpha,\vec\eta) = p(\mathcal{D}|\boldsymbol{\vec l},\vec\eta)p(\boldsymbol{\vec l}|\vec\alpha)=p(\mathcal{D}|\boldsymbol{\vec l},\vec\eta)\cdot\prod_{d=1}^D p(\boldsymbol{\vec l}_d|\vec\alpha)
\end{align}
其中 $\boldsymbol{\vec l}=\{\boldsymbol{\vec l}_d\}_{d=1}^D=\{\boldsymbol{\vec z}^+, \boldsymbol{\vec z}\}=\{\boldsymbol{\vec z}_d^+,\boldsymbol{\vec z}_d\}_{d=1}^D$。首先,我们可以推导出
\begin{align}
p(\boldsymbol{\vec l}_d|\vec\alpha)=p(\boldsymbol{\vec z}_d|\boldsymbol{\vec z}_d^+)\int p(\boldsymbol{\vec z}_i^+|\vec\theta_d)p(\vec\theta_d|\vec\alpha)\mbox{d}\vec\theta_d=\frac{\Delta(\vec n_{P_d}+\vec\alpha)}{\Delta(\vec\alpha)}  \cdot\prod_{k=1}^K\left(\frac{n_{P_d}^{(k)}}{N_d}\right)^{n_{S_d}^{(k)}}
\end{align}
其中,$\vec n_{P_d}=\{n_{P_d}^{(k)}\}_{k=1}^K$ 和 $\vec n_{S_d}=\{n_{S_d}^{(k)}\}_{k=1}^K$。$n_{P_d}^{(k)}$ 和 $n_{S_d}^{(k)}$ 分别是第 $d$ 个伪文档和原始文档中属于第 $k$ 个主题的词的数量。在这里,我们采用了 $\Delta$ 函数,如下所示:
\begin{equation}
    \Delta(\vec\alpha)=\frac{\prod_{k=1}^K\Gamma(\alpha)}{\Gamma(\sum_{k=1}^K\alpha)}
\end{equation}
\begin{equation}
    \Delta(\vec n_{P_d}+\vec\alpha)=\frac{\prod_{k=1}^K\Gamma(n_{P_d}^{(k)}+\alpha)}{\Gamma(\sum_{k=1}^K n_{P_d}^{(k)}+\alpha)}=\frac{\prod_{k=1}^K\Gamma(n_{P_d}^{(k)}+\alpha)}{\Gamma(N_d+K\alpha)}
\end{equation}

类似的,我们可以改写 $p(\mathcal{D}|\boldsymbol{\vec l},\vec\eta)=p(\mathcal{S}|\boldsymbol{\vec z},\vec\eta)p(\mathcal{P}|\boldsymbol{\vec z}^+,\vec\eta)$ 为
\begin{align}
p(\mathcal{D}|\boldsymbol{\vec l},\vec\eta)=\prod_{i=1}^{W}\beta_{z_i}^{(w_i)}\cdot\prod_{j=1}^{W^+}\beta_{z_j^+}^{(w_j)}
=\prod_{k=1}^K\left(\prod_{\{i:z_i=k\}}\beta_k^{(w_i)}\cdot\prod_{\{j:z_j^+=k\}}\beta_k^{(w_j)}\right)
=\prod_{k=1}^K\prod_{v=1}^V(\beta_k^{(v)})^{n_k^{(v)}}
\end{align}
其中,$W$ 和 $W^+$ 分别是$\mathcal{S}$和$\mathcal{P}$中的词数,$n_k^{(v)}$ 是分配给$\mathcal{D}$中第 $k$ 个主题的词 $v$ 的出现次数。然后,通过对$\vec\beta$积分,我们可以得到
\begin{equation}
    p(\mathcal{D}|\boldsymbol{\vec l},\vec\eta)=\prod_{k=1}^K\frac{\Delta(\vec n_k+\vec\eta)}{\Delta(\vec\eta)}
\end{equation}
\begin{equation}
    \Delta(\vec\eta)=\frac{\prod_{v=1}^K\Gamma(\eta)}{\Gamma(\sum_{v=1}^V\eta)}
\end{equation}
\begin{equation}
    \Delta(\vec n_k+\vec\eta)=\frac{\prod_{v=1}^V\Gamma(n_k^{(v)}+\eta)}{\Gamma(\sum_{v=1}^V n_k^{(v)}+\eta)}=\frac{\prod_{v=1}^V\Gamma(n_k^{(v)}+\eta)}{\Gamma(n_k+V\eta)}
\end{equation}
其中$\vec n_k=\{n_k^{(v)}\}_{v=1}^V$,且$n_k=\sum_{v=1}^V n_k^{(v)}$。
现在联合概率分布 Eq.(\ref{joint1}) 变成:
\begin{align}
\label{joint2}
p(\mathcal{D},\boldsymbol{\vec l}|\vec\alpha,\vec\eta)=\prod_{k=1}^K\frac{\Delta(\vec n_k+\vec\eta)}{\Delta(\vec\eta)}
\cdot \prod_{d=1}^D \left[\frac{\Delta(\vec n_{P_d}+\vec\alpha)}{\Delta(\vec\alpha)}\cdot\prod_{k=1}^K\left(\frac{n_{P_d}^{(k)}}{N_d}\right)^{n_{S_d}^{(k)}}\right]
\end{align}

接下来需要求解两个条件后验概率分布:(1)为伪文档$P_d$中的词$w_{d,n}^+$采样一个主题$z_{d,n}^+$的条件后验概率分布;(2)对于原始文档$S_d$中的词$w_{d,n}$,将采样一个主题$z_{d,n}$条件后验概率分布。
对于(1),我们有

\begin{equation}
\begin{aligned}
    &p(z_{d,n}^+=k|\boldsymbol{\vec l}_{\neg(P_{d,n})},\mathcal{D})=\frac{p(\boldsymbol{\vec l},\mathcal{D})}{p(\boldsymbol{\vec l}_{\neg(P_{d,n})},\mathcal{D})}\propto\frac{p(\boldsymbol{\vec l},\mathcal{D})}{p(\boldsymbol{\vec l}_{\neg(P_{d,n})},\mathcal{D}_{\neg(P_{d,n})})}\\
    &=\frac{\Delta(\vec n_k+\vec\eta)}{\Delta(\vec n_{k,\neg(P_{d,n})} + \vec\eta)}\cdot\frac{\Delta(\vec n_{P_d}+\vec\alpha)}{\Delta(\vec n_{P_d,\neg(P_{d,n})}+\vec\alpha)}\cdot\prod_{j=1}^K\left(\frac{N_d-1}{N_d}\cdot\frac{n_{P_d}^{(j)}}{n_{P_d,\neg(P_{d,n})}^{(j)}}\right)^{n_{S_d}^{(j)}}\\
    &\propto\frac{n_{k,\neg(P_{d,n})}^{(v)}+\eta}{\sum_{i=1}^V(n_{k,\neg(P_{d,n})}^{(i)}+\eta)}
    \cdot\frac{n_{P_d,\neg(P_{d,n})}^{(k)}+\alpha}{N_d-1+K\alpha}\cdot\left(\frac{N_d-1}{N_d}\cdot\frac{n_{P_d,\neg(P_{d,n})}^{(k)}+1}{n_{P_d,\neg(P_{d,n})}^{(k)}}\right)^{n_{S_d}^{(k)}}
    \end{aligned}
\end{equation}
其中$\boldsymbol{\vec l}=\{\boldsymbol{\vec l}_d\}_{d=1}^D=\{\boldsymbol{\vec z}^+, \boldsymbol{\vec z}\}=\{\boldsymbol{\vec z}_d^+,\boldsymbol{\vec z}_d\}_{d=1}^D$。$n_{P_d}^{(k)}$ 和 $n_{S_d}^{(k)}$ 分别是第 $d$ 个伪文档和原始文档中属于第 $k$ 个主题的词的数量。而 $n_k^{(v)}$ 是分配给 $\mathcal{D}$ 中第 $k$ 个主题的词 $v$ 的出现次数。所有带有 $\neg\bullet$ 的计数表示排除来自 $\bullet$ 的计数。类似地,对于(2),原始文档$S_d$中的词$w_{d,n}$,其采样一个主题$z_{d,n}$的条件后验概率分布为公式为:

\begin{equation}
\begin{aligned}
&p(z_{d,n}=k|\boldsymbol{\vec l}_{\neg(S_{d,n})},\mathcal{D})=\frac{p(\boldsymbol{\vec l},\mathcal{D})}{p(\boldsymbol{\vec l}_{\neg(S_{d,n})},\mathcal{D})}\propto\frac{p(\boldsymbol{\vec l},\mathcal{D})}{p(\boldsymbol{\vec l}_{\neg(S_{d,n})},\mathcal{D}_{\neg(S_{d,n})})}\\
&=\frac{\Delta(\vec n_k+\vec\eta)}{\Delta(\vec n_{k,\neg(S_{d,n})} + \vec\eta)}\cdot\frac{\Delta(\vec n_{P_d}+\vec\alpha)}{\Delta(\vec n_{P_d}+\vec\alpha)}\cdot\prod_{j=1}^K\left(\frac{n_{P_d}^{(j)}}{N_d}\right)^{n_{S_d}^{(j)}}/\left(\frac{n_{P_d}^{(j)}}{N_d}\right)^{n_{S_d,\neg(S_{d,n})}^{(j)}}\\
&\propto\frac{n_{k,\neg(S_{d,n})}^{(v)}+\eta}{\sum_{i=1}^V(n_{k,\neg(S_{d,n})}^{(i)}+\eta)}\cdot\left(\frac{n_{P_d}^{(k)}}{N_d}\right)
\end{aligned}
\end{equation}