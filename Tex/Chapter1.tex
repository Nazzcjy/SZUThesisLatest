\chapter{绪论}\label{chap:intro}
\markboth{第一章\ \ 绪论}{}
\section{研究背景}\label{sec:background}

当前模板完美遵循《学术学位硕士学位论文印刷格式样式》与《专业硕士学位论文印刷格式样式》中规定的学位论文撰写要求和封面设定。
目前支持 Windows 操作系统(Linux、MacOS可能会有未知问题);目前仅支持 Xe\LaTeX{} 引擎;文献编译引擎biber (biblatex)。
支持中文书签、中文渲染、拷贝 PDF 中的文本到其他文本编辑器等特性。


\section{系统要求}\label{sec:system}

szuthesis 宏包可以在目前主流的 \href{https://en.wikibooks.org/wiki/LaTeX/Introduction}{\LaTeX{}} 发行版中使用,
如 \TeX{}Live 和 MiK\TeX{}。因 C\TeX{} 套装已停止维护,\textbf{请勿使用}。
请勿混淆 C\TeX{} 套装\footnote{\url{http://www.ctex.org/CTeX}}与 C\TeX{} 宏集\footnote{\url{https://ctan.org/pkg/ctex?lang=en}}。
C\TeX{} 套装基于 Windows 下的 MiKTeX 开发,在其基础上增加了对中文的完整支持,已于 2012 年起停止维护。
而 C\TeX{} 宏集是通用 \LaTeX{} 排版框架,为中文 \LaTeX{} 文档提供了汉字支持,主要包括宏包 ctex 以及中文文档类 ctexart、 ctexbook 等。

推荐的 \LaTeX{} 发行版如下:

\begin{center}
    %\footnotesize% fontsize
    %\setlength{\tabcolsep}{4pt}% column separation
    %\renewcommand{\arraystretch}{1.5}% row space 
    \begin{tabular}{lc}
        \toprule
        操作系统         & \LaTeX{}发行版                                                                                        \\
        \midrule
        Linux 或 Windows & \href{https://www.tug.org/texlive/}{\TeX{}Live Full} 或 \href{https://miktex.org/download}{MiK\TeX{}} \\
        MacOS            & \href{https://www.tug.org/mactex/}{Mac\TeX{} Full} 或 \href{https://miktex.org/download}{MiK\TeX{}}   \\
        \bottomrule
    \end{tabular}
\end{center}

请从各软件官网下载安装程序,勿使用不明程序源。若系统原带有旧版的 \LaTeX{} 发行版并想安装新版,请\textbf{先完全卸载旧版再安装新版}。
推荐安装2019年后的版本。可能因为网络问题导致安装速度较慢,推荐在安装时无需选择额外宏包,
安装完成后添加清华源\footnote{\url{https://mirrors.tuna.tsinghua.edu.cn/help/CTAN/}},再继续安装所需宏包。

如选择部分安装,请安装后检测以下宏包知否安装,若未安装导致的BUG不易排查:

\begin{center}
    % \small% fontsize
    \renewcommand{\arraystretch}{0.8}% row space 
    \begin{tabular}{ll}
        \toprule
        宏包                 & 功能                    \\
        \midrule
        xits                 & 开源Times New Roman字体 \\
        biber                & biblatex引擎            \\
        biblatex-gb7714-2015 & biblatex格式            \\
        latexmk              & 自动编译latex文档       \\
        \bottomrule
    \end{tabular}
\end{center}

安装 \LaTeX{} 发行版后,即可使用任意编辑器开始书写。
在这里推荐使用 VS Code\footnote{\url{https://code.visualstudio.com/}} 作为编辑器。一方面其可单纯的作为编辑器使用,
同时又可以搭配插件进行扩展。可以搭配 Git 进行版本控制,又可以安装 LaTeX Workshop 插件直接进行编译。
LaTeX Workshop 插件提供了诸如 Linting,Formatting,Intellisense,PDF 文件预览,公式预览,
全文大纲等诸多功能\footnote{\url{https://github.com/James-Yu/LaTeX-Workshop\#features-taster}}。
使用快捷键可以极大提高编写效率,例如使用 \lstinline!Ctrl+Alt+j! 可以快速从 tex 文本跳转到 PDF 中对应的位置,
而在PDF预览中使用 \lstinline!Ctrl+鼠标左键! 就可以快速定位对应的 tex 文本。


\section{编译}

\begin{enumerate}[wide=\parindent]
    \item 安装软件:根据所用操作系统和章节~\ref{sec:system} 中的信息安装 \LaTeX{} 编译环境。

    \item 获取模板:下载 szuthesis 项目。szuthesis 不仅提供了相应的模板,同时也提供了编译样例,
          下载时推荐下载整个 szuthesis 文件夹,而不是单独的 cls 文档类。

    \item 编译模板:参考项目主页编译部分。
\end{enumerate}

编译完成后即可获得这份说明文档。而这也完成了学习使用 szuthesis 撰写论文的一半进程。


\section{文档目录简介}

\subsection{Thesis.tex}

Thesis.tex 为主文档,包含了论文全篇的主要架构。其中,document 中的所有内容均可注释后避免其参与编译,包含 maketitle 等命令,
注释后可能导致章节序号发生错误,无需担心,全文编译后即可恢复。注释后可加快编译速度,例如参考文献页无须随文档实时编译,
只需要全文完成后编译参考文献页即可,这也是使用 \LaTeX{} 编写文档的优点之一。

\subsection{Temp文件夹}

编译后,生成的临时文件皆存于Temp文件夹内,包括编译得到的 PDF 文档,其存在是为了保持工作空间的整洁,因为好的心情是很重要的。

\subsection{szuthesis.cls}

\verb!texmf\tex\latex\szuthesis\szuthesis.cls! 目录下的 szuthesis.cls 为文档类,定义了论文的核心格式,
包括论文排版、引用格式、页眉页脚、字体字号等。其中,根据《印刷格式样式》规定,参考文献后的字号均与正文字号不同。

\subsection{config.tex}

szuthesis.cls 需要传入一些参数用来生成封面信息,config.tex 可用来传入这些参数。
后边则定义了一些可选的宏包,这些宏包并不完全属于《印刷格式样式》规定的排版,可以自由选择是否启用。
例如数学公式的字体、代码片段、超链接等,均在 config.tex 进行了定义,这些可以根据需要对它进行调整。

\subsection{Tex文件夹}

文件夹内为论文的所有正文内容,这也是使用 szuthesis 撰写学位论文时,主要关注和修改的一个位置。
\textbf{注:所有文件都必须采用 UTF-8 编码,否则编译后将出现乱码文本},详细分类介绍如下:

\begin{itemize}
    \item Abstract.tex:摘要信息。
    \item ChapterX.tex:论文的各个章节,可根据需要添加和撰写。\textbf{添加新章节时,注意编码格式,可拷贝一个已有的章文件再重命名,以继承文档的 UTF-8 编码}。
    \item Appendix.tex:附录,注意附录字号与正文不同,仅用于添加补充信息,如有整段文本建议放置于正文中。
    \item Acknowledgements.tex:致谢。
    \item Publications.tex: 研究成果。
\end{itemize}

\subsection{Image文件夹}

用于放置论文中所需要的图形类文件,支持格式有:jpg, png, pdf 等,需要更多支持格式可在 config.tex 中配置。
不建议为各章节图片建子目录,即使图片众多,若命名规则合理,图片查询亦是十分方便。

\subsection{Biblio文件夹}

ref.bib 为参考文献信息,可在 config.tex 中配置。

\subsection{.vscode文件夹}

这一文件夹用于保存 VS Code 的配置文件,其中 settings.json 保存了部分 latexmk 所需的配置项。



\section{帮助与问题反馈}\label{sec:help}

对于 \LaTeX{} 相关问题,推荐使用 texdoc 命令查阅相关文档。
例如安装 lshort-chinese 宏包后,可使用 \textbf{texdoc lshort-chinese} 命令打开一份教程,包含了 \LaTeX{} 入门相关的知识。
使用 \textbf{texdoc ctex} 则可打开 ctex 宏包的文档,包含中文排版相关的内容,例如第5节中则详细介绍了中文字体字号。
大多数宏包都提供了非常详尽的文档,都可以使用 texdoc 查阅。

欢迎各位同学提出宝贵意见,一起不断改进模板。
